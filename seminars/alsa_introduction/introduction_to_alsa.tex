% small.tex
\documentclass[CJK]{beamer}
%\usepackage{fontspec}
\usepackage{xeCJK}
%\setmainfont{Microsoft YaHei}
\setCJKmainfont{Microsoft YaHei}
%\usetheme{default}
\usetheme{Boadilla}
\usepackage{graphicx}

\usepackage{tikz}
\usetikzlibrary{shapes,arrows,decorations.pathmorphing,backgrounds,positioning,fit,petri}

\title[ALSA音频系统介绍]{ALSA音频系统介绍}
\author{Raymond Wen}


\begin{document}

\begin{frame}
    \titlepage
\end{frame}

\begin{frame}
    \frametitle{大纲}
    \tableofcontents[pausesections]
\end{frame}

\section{ALSA的发展历史}
\begin{frame}[t]
    \frametitle{历史}
    \begin{block}{}
        \begin{itemize}
            \item ALSA: Advanced Linux Sound Architecture 
            \item OSS: Open Sound System
        \end{itemize}
    \end{block}
    \begin{figure}[b]
        %\includegraphics[width=\textwidth]{"history.png"}
        \includegraphics[height=160bp]{"history.png"}
    \end{figure}
\end{frame}

\section{ALSA的结构}
\begin{frame}[t]
    \frametitle{Linux 中的音频系统层次 - 架构}
    \begin{figure}
        \includegraphics[height=200bp]{"linux_audio_layers.png"}
    \end{figure}
\end{frame}

\begin{frame}
    \frametitle{ALSA的特点 - 架构}
    \begin{itemize}
        \item 对声卡的支持广泛
        \item 模块化的结构
        \item 线程安全的设计,支持多核处理器
        \item 友好的用户接口
        \item 兼容OSS接口
    \end{itemize}
\end{frame}

\begin{frame}
    \frametitle{ALSA的组成 - 架构}
    \begin{columns}
        \begin{column}{0.4\textwidth}
            \begin{itemize}
                \item ALSA 设备驱动
                    \begin{itemize}
                        \item machine driver
                        \item platform driver
                        \item codec driver
                    \end{itemize}
                \item ALSA 库: asound.so
                \item ALSA 工具
                    \begin{itemize}
                        \item amixer
                        \item aplay
                        \item arecord
                    \end{itemize}
            \end{itemize}
        \end{column}
        \begin{column}{0.5\textwidth}
            \begin{figure}
                \includegraphics[height=150bp]{"alsa_components_layer.png"}
            \end{figure}
        \end{column}
    \end{columns}
\end{frame}

\begin{frame}[fragile]
    \frametitle{ALSA驱动设备结构 - 驱动}
    \begin{figure}
        \includegraphics[height=120bp]{"alsa_devices.png"}
    \end{figure}
    \begin{verbatim}
    ls /dev/snd
        controlC0 pcmC0D0c pcmC0D0p timer
    \end{verbatim}
\end{frame}

\begin{frame}
    \frametitle{ALSA中音频数据的结构 - 驱动}
    \begin{columns}
        \begin{column}{0.3\textwidth}
            \begin{itemize}
                \item Buffer 
                \item Period
                \item Frame
                \item Sample
            \end{itemize}
        \end{column}
        \begin{column}{0.5\textwidth}
            \begin{figure}
                \includegraphics[height=150bp]{"alsa_data.jpg"}
            \end{figure}
        \end{column}
    \end{columns}
\end{frame}

\begin{frame}
    \frametitle{音频数据示例 - 驱动}

    \begin{alertblock}{}
        假定需要播放 stereo, 16bit, 44.1 KHz 的音频流
    \end{alertblock}
    \pause
    \begin{exampleblock}{}
        \begin{itemize}
            \item 'stereo' = 2通道
            \item sample 大小 = 16 bits = 2 bytes
            \item frame 大小 = 2通道*sample大小 = 4 bytes
            \item 每毫秒 frame 数 = 44100/1000 = 44.1 
            \item period 大小 = 20毫秒*44.1*4 = 3528 bytes
            \item buffer 大小 = 4*3528 = 14112
            \item bps = 4*44100 = 176400 bytes/sec
        \end{itemize}
    \end{exampleblock}
\end{frame}

\section{IPVDP中的应用}
\begin{frame}
    \frametitle{IPVDP中的硬件结构 - 驱动}
    \begin{figure}
        \includegraphics[height=120bp]{"hw_arch.jpg"}
    \end{figure}
\end{frame}

\begin{frame}[t]
    \frametitle{IPVDP中音频驱动的结构 - 驱动}
    \begin{figure}
        \includegraphics[height=200bp]{"architecture.png"}
    \end{figure}
\end{frame}

\begin{frame}[t]
    \frametitle{IPVDP中音频驱动的结构 - 驱动}
    \begin{figure}
        \includegraphics[height=200bp]{"architecture_2.png"}
    \end{figure}
\end{frame}

\begin{frame}
    \frametitle{添加控制节点 - 驱动}
    \begin{itemize}
        \item 定义控制节点 struct snd\_kcontrol\_new
        \item 添加snd\_kcontrol\_new到snd\_soc\_codec
        \item 利用预定义的宏
    \end{itemize}
\end{frame}

\begin{frame}[fragile]
    \frametitle{定义snd\_kcontrol\_new - 驱动}
    \begin{verbatim}
    static struct snd_kcontrol_new my_control __devinitdata = {  
        .iface = SNDRV_CTL_ELEM_IFACE_MIXER,  
        .name = "PCM Playback Switch",  
        .index = 0,  
        .access = SNDRV_CTL_ELEM_ACCESS_READWRITE,  
        .private_value = 0xffff,  
        .info = my_control_info,  
        .get = my_control_get,  
        .put = my_control_put  
    };

    snd_ctl_add(card, my_control);
    //name 属性很重要,是ALSA用来分类的依据
    //     命名规则为: 源 方向 功能
    \end{verbatim}
\end{frame}

\begin{frame}[fragile]
    \frametitle{通过宏添加控制节点 - 驱动}
    \begin{verbatim}
static const struct snd_kcontrol_new cx20709_snd_controls[]= {
  SOC_DOUBLE_R_TLV("PCM Playback Volume", DAC1_GAIN_LEFT,	
      DAC2_GAIN_RIGHT, 0, 0x4F, 0, db_scale),
  SOC_SINGLE(  "Lineout Playback Volume", DAC3_GAIN_SUB,		 
      0, 0xBF, 0),
  SOC_SINGLE_BOOL_EXT("Mic Stream Switch", 
      (unsigned long)&dsp_stream_mic_enable, 
      cx20709_stream_get, cx20709_stream_put),
  SOC_ENUM_EXT("Lineout Source Select", input_source_enum[0],
      cx20709_input_source_get, cx20709_input_source_put),
}

snd_soc_add_controls(codec, cx20709_snd_controls, 
  ARRAY_SIZE(cx20709_snd_controls))

    \end{verbatim}
\end{frame}

\begin{frame}[fragile]
    \frametitle{应用程序使用控制节点 - 驱动}
    \begin{verbatim}

    amixer sset 'PCM' 80%
    amixer sset 'Lineout' 60%
    amixer sset 'MIC Stream' off
    amixer sset 'Lineout Source Select' 'Scale_out'

    \end{verbatim}
\end{frame}

\begin{frame}
    \frametitle{音频通道设置 - 架构}
        \includegraphics[height=200bp]{"ipvdp_sys_arch.png"}
\end{frame}

\begin{frame}
    \frametitle{音频通道设置 - 驱动}
        \includegraphics[height=120bp]{"hw_arch.jpg"}
\end{frame}

\begin{frame}
    \frametitle{音频通道设置 - 驱动}
    \includegraphics[height=50bp]{"audio_path_scene1.png"}
    \newline
    \includegraphics[height=120bp]{"audio_path_normal.png"}
\end{frame}

\begin{frame}
    \frametitle{音频通道设置 - 驱动}
    \includegraphics[height=50bp]{"audio_path_scene2.png"}
    \newline
    \includegraphics[height=120bp]{"audio_path_main_dc.png"}
\end{frame}

\begin{frame}
    \frametitle{音频通道设置 - 驱动}
    \includegraphics[height=50bp]{"audio_path_scene3.png"}
    \newline
    \includegraphics[height=120bp]{"audio_path_subphone.png"}
\end{frame}

\begin{frame}[fragile]
    \frametitle{切换音频通道 - 应用}
    \begin{verbatim}
    amixer sset 'Class D power' off
    amixer sset 'Mic AGC' on
    amixer sset 'AEC1 Tail Length' 32
    amixer sset 'EQ' on
    amixer sset 'LEC' off
    amixer sset 'Stream Mixer Mute' off
    amixer sset 'Linein ADC Mute' off
    amixer sset 'Lineout DAC Mute' off
    amixer sset 'Speaker DAC Mute' off
    amixer sset 'Mic ADC Mute' off
    amixer sset 'Voicein Source Select' 'Stream2_out'
    amixer sset 'Speaker Source Select' 'Scale_out'
    \end{verbatim}
\end{frame}

\begin{frame}[fragile]
    \frametitle{切换音频通道 - 应用}
    \begin{verbatim}
    提供命令切换:
        audio_path 0|1|2

    提供api调用切换:
        setAudioPathMode(mode);
    \end{verbatim}
\end{frame}

\begin{frame}
    \frametitle{耳机插入事件检测 - 驱动}
    \includegraphics[height=140bp]{"jack_detection.png"}
    示例: soc-jack.c from alsa driver
\end{frame}

\begin{frame}[fragile]
    \frametitle{ALSA应用程序的流程 - 应用}
    \begin{verbatim}
      open_the_device();
      set_the_parameters_of_the_device();
      while (!done) {
           /* one or both of these */
           receive_audio_data_from_the_device();
           deliver_audio_data_to_the_device();
      }
      close the device
    \end{verbatim}
\end{frame}

\begin{frame}[fragile]
    \frametitle{ALSA应用程序的流程 - 应用}
    \begin{verbatim}
      snd_pcm_open(&handle, "default", 
          SND_PCM_STREAM_PLAYBACK, 0);
      snd_pcm_hw_params_alloca(&params);
      snd_pcm_hw_params_set_format();
      snd_pcm_hw_params_set_channels();
      snd_pcm_hw_params_set_rate_near();
      snd_pcm_hw_params_set_period_time_near();
      snd_pcm_hw_params_set_buffer_time_near();
      snd_pcm_hw_params(handle, params);
      while (!done) {
          snd_pcm_write(handle, data, count);
      }
      snd_pcm_drain(handle);
      snd_pcm_close(handle);
    \end{verbatim}
    示例: aplay.c from alsa utils
\end{frame}

\begin{frame}[t]
    \frametitle{混音的实现 - 应用}
    \begin{alertblock}{需求}
        \begin{columns}
            \begin{column}{0.7\textwidth}
                \begin{itemize}
                    \item 同时播放两路以上的数字音频(混音)
                    \item 使用默认设备会得到音频设备冲突的错误
                \end{itemize}
            \end{column}
            \begin{column}{0.3\textwidth}
                \includegraphics[height=40bp]{"mix.png"}
            \end{column}
        \end{columns}
    \end{alertblock}
    \pause
    \begin{exampleblock}{dmix插件}
        \begin{itemize}
            \item ALSA支持插件
            \item 插件生成虚拟设备
            \item 指定使用dmix插件的虚拟设备: plug:dmix
            \begin{itemize}
                \item snd\_pcm\_open(\&handle, "plug:dmix", SND\_PCM\_STREAM\_PLAYBACK, 0)
                \item 
                \item aplay -D "plug:dmix" test.wav
            \end{itemize}
        \end{itemize}
    \end{exampleblock}
\end{frame}

\begin{frame}
    \frametitle{总结}
    \begin{itemize}
        \item ALSA的架构
        \item ALSA驱动的组成
        \item 添加控制节点 
        \item ALSA应用程序的结构
    \end{itemize}
\end{frame}

\end{document}
